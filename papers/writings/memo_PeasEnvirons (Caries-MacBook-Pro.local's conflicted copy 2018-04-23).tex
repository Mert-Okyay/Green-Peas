\documentclass[12pt]{article}
%\documentclass[useAMS,usenatbib]{mn2e}
%\documentclass[apj]{emulateapj}

%%% TODO:
%1. fix numbers on numebrs.pro
%2. add table 1 and 2 (frac of barred galaxies and fraction fo galaxies barred
%3. fix figure 1.
%4 write results section
%5 write discussion section - add figute of 4 plots conclusions,etc.

%\voffset-1.25cm
\newcommand\ion[2]{#1$\;${\scshape{#2}}}%                       % ion, i.e., CII = \ion{C}{ii}
%\setlength{\textwidth}{6.5in} 
%\setlength{\textheight}{8.5in}
%\setlength{\topmargin}{-0.0625in} 
%\setlength{\oddsidemargin}{0in}
%\setlength{\evensidemargin}{0in} 
%\setlength{\headheight}{0in}
%\setlength{\headsep}{0in} 
%\setlength{\hoffset}{0in}
%\setlength{\voffset}{0in}



\usepackage{graphicx,natbib,times}
\usepackage{deluxetable} 
\usepackage{url}

%\date{In Prep, version April 09}
\newcommand\aj{{AJ}}
\newcommand\araa{{ARA\&A}}
\newcommand\apj{{ApJ}}
\newcommand\apjl{{ApJ}}
\newcommand\apjs{{ApJS}}
\newcommand\aap{{A\&A}}
\newcommand\nat{{Nature}}
\newcommand\mnras{{MNRAS}}
\newcommand\pasp{{PASP}}

%\pagerange{\pageref{firstpage}--\pageref{lastpage}} \pubyear{2009}

%\def\LaTeX{L\kern-.36em\raise.3ex\hbox{a}\kern-.15em
%    T\kern-.1667em\lower.7ex\hbox{E}\kern-.125emX}

%\newtheorem{theorem}{Theorem}[section]

\begin{document}
%\title{Galaxy Zoo: Environment of Peas }
\noindent {\LARGE Galaxy Zoo: The Environments of the Peas}

%\label{firstpage}

%\maketitle

\begin{abstract}
\end{abstract}
%\begin{keywords}
% galaxies: evolution, galaxies: formation, galaxies: starburst, galaxies: dwarf, galaxies: high-redshift, galaxies: Seyfert
%\end{keywords}


\section{Introduction}
\label{sec:intro}

GP BG
 
Low mass starforming galaxies are thought to be the building blocks of galaxies, playing an important role in early galaxy assembly and evolution \citep{Pillepich2015}.

Some recent far-UV studies of extreme star formation in dwarf galaxies suggest that the escape fraction of ionizing radiation from these galaxies could be the source of the unknown re-ionization of the intergalactic medium by redshift $z\sim6$ \citep{erb2016,Izotovetal2016}
However, studies of higher redshift star forming galaxies suggest that this radiation may not be sufficient for reionization \citep{Rutkowski2017,Grazian2017, Rutkowski2016}
%The intergalactic medium is composed of ionized gas and forms most of the ordinary matter in the Universe.  
%While the source of this radiation is unknown, studies of extreme star-formation in dwarf-galaxies suggest an answer to the re-ionization of this gas by suggesting that photos can escape 

`Green Peas' or `Peas' were galaxies first discovered in the Galaxy Zoo Survey, due to their small and green appearance in the SDSS images.
Followup studies have shown them to be examples of relatively lower-mass, compact, highly starforming galaxies, perhaps analogous to SF episodes occurring in the early universe.
The question we wish to investigate is do the peas have a 'typical' environment

Encoded in the large scale structure of the Universe, is a variety of cosmological parameters as well asn key infomraiton about the physical processes which underpin the formation of cosmic structures.
Heirchrical structure formation, our current paradigm of galaxy formation, places they a key role on gravititaion evolution of dark matter clustering around intial peaks providing potential werlls for gas halows and galxies to form in.


\section{Data}
New peas from SDSS

LRGs from SDSS
\section{Analysis}
CCFs are cool

%\bibliographystyle{mnras}
%\bibliography{refs}

\begin{thebibliography}{99}
\bibitem[Erb(2016)]{erb2016} Erb, D.~K.\ 2016, \nat, 529, 159 
\bibitem[Izotov et al.(2016)]{Izotovetal2016} Izotov, Y.~I., Orlitov{\'a}, I., Schaerer, D., et al.\ 2016, \nat, 529, 178 
\bibitem[Grazian et al.(2017)]{Grazian2017} Grazian, A., Giallongo, E., Paris, D., et al.\ 2017, \aap, 602, A18 
\bibitem[Pillepich et al.(2015)]{Pillepich2015} Pillepich, A., Madau, P., \& Mayer, L.\ 2015, \apj, 799, 184 
\bibitem[Rutkowski et al.(2016)]{Rutkowski2016} Rutkowski, M.~J., Scarlata, C., Haardt, F., et al.\ 2016, \apj, 819, 81 
\bibitem[Rutkowski et al.(2017)]{Rutkowski2017} Rutkowski, M.~J., Scarlata, C., Henry, A., et al.\ 2017, \apjl, 841, L27 



%\bibitem[\protect\citeauthoryear{Author}{2012}]{Author2012}
%Author A.~N., 2013, Journal of Improbable Astronomy, 1, 1
%\bibitem[\protect\citeauthoryear{Others}{2013}]{Others2013}
%Others S., 2012, Journal of Interesting Stuff, 17, 198
\end{thebibliography}


%\bsp	% typesetting comment
\label{lastpage}
\end{document}
